\documentclass[12pt]{article}
\usepackage[spanish]{babel}
\usepackage{graphicx}
\usepackage{listings}
\usepackage{color}
\usepackage{amsmath}
\usepackage{geometry}
\usepackage{hyperref}

\title{Implementación de Algoritmos Gráficos Básicos}
\author{Julio Enrique Viche Castillo}
\date{\today}

\hypersetup{
    colorlinks=true,
    urlcolor=blue,
    pdfborder={0 0 0}
}

\begin{document}
\maketitle

\begin{center}
\href{https://github.com/JulioViche/GraphicAlgorithms}{Repositorio del Proyecto en GitHub}
\end{center}

\section{Introducción}
Este trabajo presenta la implementación de algoritmos gráficos clásicos (DDA, Bresenham y FlowFill) en una aplicación Windows Forms moderna. El sistema desarrollado combina la precisión matemática de estos algoritmos con características contemporáneas como programación asíncrona y una interfaz gráfica interactiva, permitiendo a los usuarios visualizar y controlar el proceso de trazado de líneas, círculos y relleno de áreas.

\section{Descripción de la Interfaz}

\subsection{Componentes Principales}
La interfaz gráfica se compone de:

\begin{itemize}
    \item \textbf{Canvas}: PictureBox de 400x400 píxeles
    \item \textbf{Panel de Control}: Agrupado por secciones
\end{itemize}

\subsection{Secciones de Control}

\subsubsection{Dibujo de Líneas}
\begin{itemize}
    \item Campos de entrada: $(x_0,y_0)$ y $(x_1,y_1)$
    \item Botón Draw DDA
    \item Botón Draw Bresenham
\end{itemize}

\subsubsection{Dibujo de Círculos}
\begin{itemize}
    \item Campos de entrada: centro $(cx,cy)$ y radio $(r)$
    \item Botón Circle Bresenham
\end{itemize}

\subsubsection{Dibujo de Rombo}
\begin{itemize}
    \item Campos de entrada: ancho y alto
    \item Botón Draw Rhombus
\end{itemize}

\subsection{Controles Generales}
\begin{itemize}
    \item Checkbox Animations Enabled
    \item Botón Cancel Animation
    \item Botón Clear Canvas
\end{itemize}

\subsection{Herramienta de Relleno}
\begin{itemize}
    \item Activación mediante click en el canvas
    \item Estado controlado por variable fillEnabled
    \item Color de relleno predefinido (Negro)
\end{itemize}

\subsection{Estados de la Interfaz}
\subsubsection{Estado Normal}
\begin{itemize}
    \item Todos los botones habilitados
    \item Relleno disponible
    \item Canvas receptivo a clicks
\end{itemize}

\subsubsection{Estado de Animación}
\begin{itemize}
    \item Botones de dibujo deshabilitados
    \item Solo Cancel Animation habilitado
    \item Relleno temporalmente desactivado
\end{itemize}

\section{Algoritmos Implementados}

\subsection{Algoritmo DDA (Digital Differential Analyzer)}
\begin{lstlisting}
public async Task DrawDDA
(
    PictureBox canvas,
    Bitmap bitmap,
    bool animationEnabled,
    CancellationToken token
)
{
    AnimationManager am = new AnimationManager
    (
        "DDA LINE",
        animationEnabled,
        canvas,
        bitmap
    );

    int steps = Math.Max(Math.Abs(dx), Math.Abs(dy));
    float xInc = (float)dx / steps;
    float yInc = (float)dy / steps;

    float xk = start.X;
    float yk = start.Y;

    am.AlgorithmStart();

    for (int i = 0; i <= steps; i++)
    {
        if (token.IsCancellationRequested)
        {
            Console.WriteLine("Algorithm cancelled.");
            return;
        }

        await am.SetPixel
        (
            (int)Math.Round(xk),
            (int)Math.Round(yk),
            Color.Black,
            2
        );

        xk += xInc;
        yk += yInc;
    }

    canvas.Invalidate();
    am.AlgorithmEnd();
}
\end{lstlisting}

\subsection{Análisis del Algoritmo DDA}

\subsubsection{Implementación Base}
El algoritmo implementa un trazador de líneas asíncrono con los siguientes componentes:

\begin{lstlisting}
steps = Math.Max(Math.Abs(dx), Math.Abs(dy))
xInc = dx / steps
yInc = dy / steps
\end{lstlisting}

\subsubsection{Componentes Principales}
\paragraph{Gestión de Animación}
\begin{itemize}
    \item Utiliza AnimationManager para control de dibujo
    \item Parámetros:
    \begin{itemize}
        \item Nombre del algoritmo: "DDA LINE"
        \item Estado de animación
        \item Canvas y bitmap para renderizado
    \end{itemize}
\end{itemize}

\paragraph{Cálculo de Incrementos}
\begin{equation}
    xInc = \frac{dx}{steps}, \quad yInc = \frac{dy}{steps}
\end{equation}

\subsubsection{Proceso Iterativo}
Para cada paso $i$ de 0 a $steps$:
\begin{enumerate}
    \item Redondeo de coordenadas: $(x_k, y_k)$
    \item Dibujo de píxel mediante SetPixel
    \item Incremento de posición: $x_k += xInc$, $y_k += yInc$
\end{enumerate}

\subsubsection{Características Técnicas}
\paragraph{Eficiencia}
\begin{itemize}
    \item \textbf{Complejidad Temporal}: $O(steps)$
    \item \textbf{Complejidad Espacial}: $O(1)$
    \item \textbf{Operaciones por iteración}:
    \begin{itemize}
        \item 2 sumas en punto flotante
        \item 2 operaciones de redondeo
        \item 1 operación de dibujo asíncrona
    \end{itemize}
\end{itemize}

\paragraph{Características Asíncronas}
\begin{itemize}
    \item Implementación mediante async/await
    \item Control de cancelación vía CancellationToken
    \item Actualización visual mediante Invalidate()
\end{itemize}

\subsubsection{Aspectos de Implementación}
\begin{itemize}
    \item Uso de punto flotante para precisión
    \item Sistema de cancelación integrado
    \item Gestión de animaciones
    \item Actualización visual del canvas
\end{itemize}

\subsection{Algoritmo de Bresenham para Líneas}
\begin{lstlisting}
public async Task DrawBresenham
(
    PictureBox canvas,
    Bitmap bitmap,
    bool animationEnabled,
    CancellationToken token
)
{
    AnimationManager am = new AnimationManager
    (
        "BRESENHAM LINE",
        animationEnabled,
        canvas,
        bitmap
    );

    int x0 = start.X;
    int y0 = start.Y;
    int x1 = end.X;
    int y1 = end.Y;

    bool steep = Math.Abs(y1 - y0) > Math.Abs(x1 - x0);

    if (steep)
    {
        (x0, y0) = (y0, x0);
        (x1, y1) = (y1, x1);
    }

    if (x0 > x1)
    {
        (x0, x1) = (x1, x0);
        (y0, y1) = (y1, y0);
    }

    int dx = x1 - x0;
    int dy = Math.Abs(y1 - y0);
    int error = dx / 2;
    int ystep = y0 < y1 ? 1 : -1;
    int y = y0;

    am.AlgorithmStart();

    for (int x = x0; x <= x1; x++)
    {
        if (token.IsCancellationRequested)
        {
            Console.WriteLine("Algorithm cancelled.");
            return;
        }

        int drawX = steep ? y : x;
        int drawY = steep ? x : y;

        await am.SetPixel
        (
            drawX,
            drawY,
            Color.Black,
            2
        );

        error -= dy;
        if (error < 0)
        {
            y += ystep;
            error += dx;
        }
    }

    canvas.Invalidate();
    am.AlgorithmEnd();
}
\end{lstlisting}

\subsection{Análisis del Algoritmo de Bresenham}

\subsubsection{Implementación Base}
El algoritmo implementa un trazador de líneas asíncrono optimizado:

\paragraph{Preprocesamiento}
\begin{itemize}
    \item Detección de pendiente pronunciada (steep):
    \begin{equation}
        steep = |y_1 - y_0| > |x_1 - x_0|
    \end{equation}
    \item Intercambio de coordenadas si es necesario
    \item Garantiza dirección x positiva
\end{itemize}

\subsubsection{Componentes Principales}
\paragraph{Variables de Control}
\begin{itemize}
    \item $dx = x_1 - x_0$
    \item $dy = |y_1 - y_0|$
    \item $error = dx/2$ (error inicial)
    \item $ystep = \pm1$ (dirección del incremento en y)
\end{itemize}

\paragraph{Gestión de Animación}
\begin{itemize}
    \item AnimationManager con identificador "BRESENHAM LINE"
    \item Control de estado y visualización
\end{itemize}

\subsubsection{Proceso Iterativo}
Para cada $x$ de $x_0$ a $x_1$:
\begin{enumerate}
    \item Determinar coordenadas de dibujo según steep
    \item Dibujar píxel en $(drawX, drawY)$
    \item Actualizar error: $error -= dy$
    \item Si $error < 0$:
    \begin{itemize}
        \item Incrementar y: $y += ystep$
        \item Corregir error: $error += dx$
    \end{itemize}
\end{enumerate}

\subsubsection{Características Técnicas}
\paragraph{Eficiencia}
\begin{itemize}
    \item \textbf{Complejidad Temporal}: $O(dx)$
    \item \textbf{Complejidad Espacial}: $O(1)$
    \item \textbf{Operaciones por iteración}:
    \begin{itemize}
        \item 1-2 operaciones enteras
        \item 1 comparación
        \item 1 operación de dibujo asíncrona
    \end{itemize}
\end{itemize}

\paragraph{Características Asíncronas}
\begin{itemize}
    \item Implementación mediante async/await
    \item Soporte para cancelación
    \item Actualización visual del canvas
\end{itemize}

\subsubsection{Optimizaciones}
\begin{itemize}
    \item Uso exclusivo de aritmética entera
    \item Manejo de casos especiales (steep)
    \item Normalización de dirección x
    \item Error inicial optimizado ($dx/2$)
\end{itemize}

\subsection{Algoritmo de Bresenham para Círculos}
\begin{lstlisting}
public async Task DrawBresenham
(
    PictureBox canvas,
    Bitmap bitmap,
    bool animationEnabled,
    CancellationToken token
)
{
    AnimationManager am = new AnimationManager
    (
        "BRESENHAM CIRCLE",
        animationEnabled,
        canvas,
        bitmap
    );

    int x = 0;
    int y = radius;
    int d = 3 - 2 * radius;

    am.AlgorithmStart();

    while (x <= y)
    {
        if (token.IsCancellationRequested)
        {
            Console.WriteLine("Algorithm cancelled.");
            return;
        }

        int xc = center.X;
        int yc = center.Y;

        await am.SetPixel(xc + x, yc + y, Color.Black, 1);
        await am.SetPixel(xc - x, yc + y, Color.Black, 0);
        await am.SetPixel(xc + x, yc - y, Color.Black, 0);
        await am.SetPixel(xc - x, yc - y, Color.Black, 0);
        await am.SetPixel(xc + y, yc + x, Color.Black, 0);
        await am.SetPixel(xc - y, yc + x, Color.Black, 0);
        await am.SetPixel(xc + y, yc - x, Color.Black, 0);
        await am.SetPixel(xc - y, yc - x, Color.Black, 0);

        if (d < 0)
        {
            d += 4 * x + 6;
        }
        else
        {
            d += 4 * (x - y) + 10;
            y--;
        }
        x++;
    }

    canvas.Invalidate();
    am.AlgorithmEnd();
}
\end{lstlisting}

\subsection{Análisis del Algoritmo de Bresenham para Círculos}

\subsubsection{Implementación Base}
El algoritmo implementa un trazador de círculos asíncrono utilizando simetría de octantes:

\paragraph{Inicialización}
\begin{itemize}
    \item Variables de control iniciales:
        \begin{align*}
            x &= 0 \\
            y &= radio \\
            d &= 3 - 2 * radio \text{ (discriminante)}
        \end{align*}
\end{itemize}

\subsubsection{Componentes Principales}
\paragraph{Gestión de Animación}
\begin{itemize}
    \item AnimationManager con identificador "BRESENHAM CIRCLE"
    \item Control asíncrono del dibujado
\end{itemize}

\paragraph{Simetría de Octantes}
Dibuja 8 píxeles simétricos para cada punto calculado:
\begin{itemize}
    \item $(x_c \pm x, y_c \pm y)$
    \item $(x_c \pm y, y_c \pm x)$
\end{itemize}

\subsubsection{Proceso Iterativo}
Mientras $x \leq y$:
\begin{enumerate}
    \item Dibujar 8 píxeles simétricos
    \item Actualizar discriminante $d$:
    \begin{itemize}
        \item Si $d < 0$: $d += 4x + 6$
        \item Si $d \geq 0$: $d += 4(x-y) + 10$ y $y--$
    \end{itemize}
    \item Incrementar $x$
\end{enumerate}

\subsubsection{Características Técnicas}
\paragraph{Eficiencia}
\begin{itemize}
    \item \textbf{Complejidad Temporal}: $O(r)$ donde r es el radio
    \item \textbf{Complejidad Espacial}: $O(1)$
    \item \textbf{Operaciones por iteración}:
    \begin{itemize}
        \item 8 operaciones de dibujo asíncronas
        \item 2-3 operaciones aritméticas enteras
        \item 1 comparación
    \end{itemize}
\end{itemize}

\paragraph{Características Asíncronas}
\begin{itemize}
    \item Implementación mediante async/await
    \item Soporte para cancelación
    \item Actualización visual del canvas
\end{itemize}

\subsubsection{Optimizaciones}
\begin{itemize}
    \item Uso exclusivo de aritmética entera
    \item Aprovechamiento de simetría octantal
    \item Minimización de cálculos por iteración
\end{itemize}

\subsection{Algoritmo de Relleno (FloodFill)}
\begin{lstlisting}
public static async Task FlowFill
(
    PictureBox canvas,
    Bitmap bitmap,
    Point start,
    Color replacementColor,
    bool animationEnabled,
    CancellationToken token
)
{
    if
    (
        start.X < 0 || start.X >= bitmap.Width ||
        start.Y < 0 || start.Y >= bitmap.Height
    )
        return;

    AnimationManager am = new AnimationManager
    (
        "FLOWFILL ALGORITHM",
        animationEnabled,
        canvas,
        bitmap
    );

    Color targetColor = bitmap.GetPixel(start.X, start.Y);

    if (targetColor.ToArgb() == replacementColor.ToArgb())
        return;

    HashSet<Point> visited = new HashSet<Point>();
    Stack<Point> pixels = new Stack<Point>();
    pixels.Push(start);

    int count = 0;

    am.AlgorithmStart();

    while (pixels.Count > 0)
    {
        if (token.IsCancellationRequested)
        {
            Console.WriteLine("Algorithm cancelled.");
            return;
        }

        Point p = pixels.Pop();

        if
        (
            p.X < 0 || p.X >= bitmap.Width ||
            p.Y < 0 || p.Y >= bitmap.Height
        )
            continue;

        if (visited.Contains(p))
            continue;

        Color currentColor = bitmap.GetPixel(p.X, p.Y);
        if (currentColor.ToArgb() != targetColor.ToArgb())
            continue;

        visited.Add(p);

        await am.SetPixel
        (
            p.X,
            p.Y,
            replacementColor,
            count++ % 100 == 0 ? 1 : 0
        );

        pixels.Push(new Point(p.X - 1, p.Y));
        pixels.Push(new Point(p.X, p.Y + 1));
        pixels.Push(new Point(p.X + 1, p.Y));
        pixels.Push(new Point(p.X, p.Y - 1));
    }

    canvas.Invalidate();
    am.AlgorithmEnd();
}
\end{lstlisting}

\subsection{Análisis del Algoritmo FlowFill}

\subsubsection{Implementación Base}
Implementa un algoritmo de relleno por inundación asíncrono con las siguientes estructuras:
\begin{itemize}
    \item HashSet<Point> para píxeles visitados
    \item Stack<Point> para píxeles pendientes
    \item Validación de límites del bitmap
\end{itemize}

\subsubsection{Componentes Principales}
\paragraph{Validaciones Iniciales}
\begin{itemize}
    \item Verificación de coordenadas dentro del bitmap
    \item Comprobación de color objetivo vs color de reemplazo
\end{itemize}

\paragraph{Estructuras de Datos}
\begin{equation}
    \begin{cases}
        visited: \text{conjunto de píxeles procesados} \\
        pixels: \text{pila de píxeles por procesar} \\
    \end{cases}
\end{equation}

\subsubsection{Proceso Iterativo}
Mientras existan píxeles en la pila:
\begin{enumerate}
    \item Extraer punto p de la pila
    \item Validar límites y estado del píxel
    \item Si es válido:
    \begin{itemize}
        \item Marcar como visitado
        \item Cambiar color
        \item Agregar vecinos (4-conectividad):
            \begin{itemize}
                \item $(x-1,y)$ - izquierda
                \item $(x,y+1)$ - abajo
                \item $(x+1,y)$ - derecha
                \item $(x,y-1)$ - arriba
            \end{itemize}
    \end{itemize}
\end{enumerate}

\subsubsection{Características Técnicas}
\paragraph{Eficiencia}
\begin{itemize}
    \item \textbf{Complejidad Temporal}: $O(w \times h)$ donde w,h son dimensiones del bitmap
    \item \textbf{Complejidad Espacial}: $O(w \times h)$ para estructuras auxiliares
    \item \textbf{Operaciones por píxel}:
    \begin{itemize}
        \item 1 operación de GetPixel
        \item 1 operación de SetPixel asíncrona
        \item 4 inserciones en la pila
        \item 1 inserción en conjunto visitado
    \end{itemize}
\end{itemize}

\paragraph{Características Asíncronas}
\begin{itemize}
    \item Control de animación cada 100 píxeles
    \item Soporte para cancelación
    \item Actualización visual del canvas
\end{itemize}

\subsubsection{Optimizaciones}
\begin{itemize}
    \item Uso de HashSet para búsqueda O(1)
    \item Validación temprana de límites
    \item Actualización visual optimizada
    \item Control de redundancia de píxeles
\end{itemize}

\subsubsection{Consideraciones de Rendimiento}
\begin{itemize}
    \item Consumo de memoria proporcional al área a rellenar
    \item Puede requerir optimización para áreas extensas
    \item La implementación iterativa es más estable que la recursiva
\end{itemize}

\section{Manual de Usuario}

\subsection{Interfaz Principal}
La aplicación presenta una interfaz gráfica con:
\begin{itemize}
    \item Canvas de dibujo (400x400 píxeles)
    \item Panel de controles para diferentes algoritmos
    \item Opciones de animación y limpieza
\end{itemize}

\subsection{Dibujo de Líneas}
\subsubsection{Entrada de Datos}
\begin{itemize}
    \item Coordenadas iniciales $(x_0,y_0)$
    \item Coordenadas finales $(x_1,y_1)$
    \item Rango válido: [0,399] para cada coordenada
\end{itemize}

\subsubsection{Algoritmos Disponibles}
\begin{enumerate}
    \item \textbf{DDA}: Botón "Draw DDA"
    \item \textbf{Bresenham}: Botón "Draw Bresenham"
\end{enumerate}

\subsection{Dibujo de Círculos}
\subsubsection{Parámetros Requeridos}
\begin{itemize}
    \item Centro $(x,y)$: Coordenadas en rango [0,399]
    \item Radio $(r)$: Valor positivo
    \item Validación: El círculo debe estar dentro del canvas
\end{itemize}

\subsection{Dibujo de Rombo}
\subsubsection{Parámetros}
\begin{itemize}
    \item Ancho: Valor positivo menor a 400
    \item Alto: Valor positivo menor a 400
    \item Centro: Automáticamente en el centro del canvas
\end{itemize}

\subsection{Herramienta de Relleno}
\begin{itemize}
    \item Activación: Click en área a rellenar
    \item Color de relleno: Negro
    \item Disponible cuando no hay animaciones en proceso
\end{itemize}

\subsection{Control de Animaciones}
\begin{itemize}
    \item \textbf{Checkbox Animations Enabled}: Activa/desactiva animaciones
    \item \textbf{Botón Cancel Animation}: Detiene la animación actual
    \item \textbf{Botón Clear Canvas}: Limpia el área de dibujo
\end{itemize}

\subsection{Estados del Sistema}
\subsubsection{Durante la Animación}
\begin{itemize}
    \item Botones de dibujo deshabilitados
    \item Relleno deshabilitado
    \item Botón de cancelación habilitado
\end{itemize}

\subsubsection{Estado Normal}
\begin{itemize}
    \item Todos los botones de dibujo habilitados
    \item Relleno habilitado
    \item Botón de cancelación deshabilitado
\end{itemize}

\subsection{Mensajes de Error}
El sistema muestra alertas cuando:
\begin{itemize}
    \item Valores fuera de rango [0,399]
    \item Radio del círculo menor o igual a cero
    \item Círculo fuera de los límites del canvas
    \item Dimensiones de rombo inválidas
\end{itemize}

\end{document}